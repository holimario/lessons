\documentclass[UTF8]{ctexart}
\usepackage{hyperref}
\usepackage{abstract}
\usepackage[margin=1in]{geometry}
\usepackage{graphicx}
\usepackage{gensymb}
\usepackage{color,xcolor}
\usepackage{listings}

\lstset{
 columns=fixed,       
 numbers=left,                                        % 在左侧显示行号
 numberstyle=\tiny\color{gray},                       % 设定行号格式
 frame=none,                                          % 不显示背景边框
 backgroundcolor=\color[RGB]{245,245,244},            % 设定背景颜色
 keywordstyle=\color[RGB]{40,40,255},                 % 设定关键字颜色
 numberstyle=\footnotesize\color{darkgray},           
 commentstyle=\it\color[RGB]{0,96,96},                % 设置代码注释的格式
 stringstyle=\rmfamily\slshape\color[RGB]{128,0,0},   % 设置字符串格式
 showstringspaces=false,                              % 不显示字符串中的空格
 language=c++,                                        % 设置语言
}

\begin{document}



\title{俄罗斯方块程序文档}
\author{2019012137  工物90  张鸿琳}
\maketitle

\newpage
\tableofcontents
\newpage

\section{大作业逻辑结构}
本个程序分为两个部分:①俄罗斯方块的实现②音乐的播放。为了实现这些功能,用到了OLED.c,handle.c和sound.c文件,其中OLED.c中主要是控制OLED的函数,handle.c中主要是控制俄罗斯方块运动的函数,而sound.c中主要是控制音乐播放的函数。具体实现思路如下。

\subsection{俄罗斯方块的程序实现}
\subsubsection{单点控制}
为了实现俄罗斯方块,首先要实现对OLED屏幕细化到单个像素的控制,因为块下落的时候是以单个像素点为单位下落的,所以依据老师提供的以8个像素点为单位的控制函数OLED\_W\_Dot,结合存储数组state,得到了单点控制函数dot,如下:
\begin{lstlisting}
void dot(unsigned char lx,unsigned char ly,char val){
	char temp=lx%8;
	int t=lx/8;
	char vall;
	if(val==1){
	switch(temp)
	{
	case 0:vall=0b00000001;break;
	case 1:vall=0b00000010;break;
	case 2:vall=0b00000100;break;
	case 3:vall=0b00001000;break;
	case 4:vall=0b00010000;break;
	case 5:vall=0b00100000;break;
	case 6:vall=0b01000000;break;
	case 7:vall=0b10000000;break;
	}
	state[ly][t]|=vall;
	OLED_W_Dot(lx,ly,state[ly][t]);
	}
	else if(val==0){
	switch(temp)
		{
		case 0:vall=0b11111110;break;
		case 1:vall=0b11111101;break;
		case 2:vall=0b11111011;break;
		case 3:vall=0b11110111;break;
		case 4:vall=0b11101111;break;
		case 5:vall=0b11011111;break;
		case 6:vall=0b10111111;break;
		case 7:vall=0b01111111;break;
		}
	state[ly][t]&=vall;
	OLED_W_Dot(lx,ly,state[ly][t]);
	}
}
\end{lstlisting}

\subsubsection{对块的操作}
实现了单点操作函数后,本程序中以2*2为单元格,而每个俄罗斯方块都由四个单元格构成,这样先实现在特定位置的单元格点亮函数ll和单元格熄灭函数dd,方便下一步实现块的生成和对块的各种操作。

块的类型共有7种,由lightened函数利用单点点亮函数实现,在固定的初始位置生成下落的块。

块的运动有:①下移②左移③右移④逆时针旋转⑤顺时针旋转。下移是自动进行的,在程序开始后,块会自动下落,而左移、右移和旋转是人为控制的,在进行某个操作前,需要对操作的合理性进行判定,所以需要实现相应的判定函数,即testmove,testright,testleft,testlrotate,testrrotate,分别用于判定是否可以下落、右移、左移、逆时针旋转、顺时针旋转,其原理是先对块的操作进行模拟,得到块在操作后的单元格坐标,再判定模拟出的坐标是否越界或者在该位置处已经存在点亮的块。

在判定某个操作合理后,该操作会被执行,对应的操作执行函数分别为move,right,left,lrotate,rrotate,其都是利用ll和dd函数,根据现在块所处的单元格位置进行变换。其中lrotate和rrotate函数比较冗杂,因为不同的块的不同姿态对应的旋转操作的实现都有所区别,只能分情况实现。如果testmove函数判定可以移动,块会不断下移。

左移、右移、旋转是人为操作实现的,所以需要利用按键中断,中断函数如下:
\begin{lstlisting}
void PORTA_IRQHandler(){
	if((GPIOA_PDIR & 0b1000000000000)==0)
	{
		if(testlrotate()==1)
		{
			lrotate();
		}

	}
	else if((GPIOA_PDIR & 0b10000000000000)==0)
	{
		if(testrrotate()==1)
		{
			rrotate();
		}
	}
	else if((GPIOA_PDIR & 0b100000000000000)==0)
	{
		if(testleft()==1)
		{
			left();
		}
	}
	else if((GPIOA_PDIR & 0b10000000000000000)==0)
	{
		if(testright()==1)
		{
			right();
		}
	}
	else if((GPIOA_PDIR & 0b1000000000000000)==0)
		{
		if(testmove()==1)
		{
			move();
		}
		if(testmove()==1)
		{
			move();
		}
		if(testmove()==1)
		{
			move();
		}
		if(testmove()==1)
		{
			move();
		}
		}
	delay();
	PORTA_PCR12|=0x01000000;
	PORTA_PCR13|=0x01000000;
	PORTA_PCR14|=0x01000000;
	PORTA_PCR15|=0x01000000;
	PORTA_PCR16|=0x01000000;
}
\end{lstlisting}

\subsubsection{分数统计以及刷新}
当一个块落到底部后,其信息会被更新到lightened数组中,表明该区域已经被之前的块占据,用于之后的操作判定中。

同时程序会执行score函数,判定是否有一层满了,由于游戏区域被限定在64*80的区域内,所以只需要对lightened数组中的某一列求和,如果和等于极限值,那么该层就满了,为了减少判定时间,所以引入了high和low,确定最后落下的块的行范围,这样求和检测操作只需要执行1-4次即可,节约了时间。

如果存在某一层满了,就会执行above函数,也就是将该层消去,同时将上面的层平移下来,利用lightened数组和ll、dd函数,这样的操作是很容易实现的,同时也是为了节约时间,引入了min变量记录堆积的最高层数。

之后会更新相应的分数,通过addscore显示出更新的分数。

之后会执行lose函数,判定是否越界,如果越界,则会显示“Game Over”,并进入新的游戏。

\subsection{音乐的播放}
音乐播放的代码较为简单,利用老师所给的代码思路即可完成。主要思路就是利用时钟中断和PWM波,通过时钟中断切换不同的音高并控制音长,而利用PWM的频率设置不同的音高。由于乐谱需要,在老师所给代码的基础上又引入了几个音高,实现了butterfly的播放。

在游戏上加上BGM是考虑到原来的主程序可以使用时钟中断和delay函数两种方式实现块的周期下落,如果使用delay函数和按键中断搭配实现游戏控制,就可以再利用时钟中断实现BGM播放,因为按键中断和时钟中断两个中断基本不会发生冲突。

\section{一些待改进处}

本次作业还有很多可以改进之处:
\begin{itemize}
\item 随机函数不够随机。由于使用了rand函数随机生成块,而rand是和系统时间相关的,这导致前面几个生成的块总是一样的,不过该影响随着游戏进行而消失
\item 由于中断发生的时间随机,所以假如在程序的move函数中发生中断,可能会导致出现一两个像素块出问题,不过在调试后基本没有发生过,理论概率也很低,对程序整体运行也没有影响
\item 在游戏设计方面,块的大小比较小,而整个游戏空间较大,以至于游戏缺少挑战性
\item 游戏的互动性还可以增强,理论上,可以使用加速度传感器控制块的移动,同时BGM可以替换为游戏音效,在某些操作时触发
\item 游戏的灵活性也还可以增强,比如可以在开始时增加选择界面,利用按键确定块的下落速度,以及游戏空间大小,还可以增加暂停键
\item 没有设置中断嵌套,所以时钟中断和按键中断嵌套的影响尚且未知
\item 代码方面,还不够精简,在编写最后发现一些函数还可以进一步合并和归纳整理
\end{itemize}

\section{借鉴部分}
\begin{itemize}
\item 负责OLED屏幕控制的OLED.c和OLED.h文件的大部分内容(除了单点控制函数)是曾老师提供的
\item 负责音乐播放控制的sound.c和sound.h文件中的初始化函数和整个代码逻辑是曾老师提供的
\end{itemize}

\end{document}
