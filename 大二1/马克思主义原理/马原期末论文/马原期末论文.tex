\documentclass[UTF8]{ctexart}
\usepackage{hyperref}
\usepackage{abstract}
\usepackage[margin=1in]{geometry}
\usepackage{graphicx}
\usepackage{gensymb}
\usepackage{fancybox}
\begin{document}

\title{最低工资规定是对劳动者的重要权益保护}
\author{2019012137  工物90  张鸿琳}
\maketitle

《最低工资法不可取》一文对最低工资法提出了质疑,认为其侵害了低薪工人阶级的权益,这个观点过于片面,最低工资法是对工人阶层重要的权益保护。

\section{工资与价值增值}
马克思指出:“工资,即资本家为劳动能力而支付的价格,对于资本家来说,是生产费用——预付的货币,这笔货币预付出去是为了赚更多的钱,只是赚钱的手段。” 

劳动者工资也就是劳动力的价格,在马克思的工资决定理论中,工资的形成与决定受到供求规律和竞争规律共同的制约和影响,买主之间的竞争将提高工资水平,而卖主之间的竞争又会降低工资水平,同时市场对劳动力商品的供需关系也会影响相应所需劳动者的工资水平。但是这些是价格的影响因素,劳动力的价格本质上是由劳动力价值决定的,而劳动力价值又由劳动力维持生活所需的生活资料所决定,并非像《最低工资法不可取》一文中所述“市场的‘供需’是劳动力价格的惟一决定因素”。

马克思认为资本的根本目的是追求价值增值,要实现价值增值,也就需要无偿地占有劳动者剩余劳动价值。要实现这个目的就必须与他人发生雇佣关系,从而可以剥削劳动者的剩余劳动价值,这种雇佣劳动是死的劳动对活的劳动的单向度的、不平等的权力支配关系。资本为了获得更多的价值增值,就会通过各种手段增加劳动者剩余劳动时间的比重。在资本家看来,工资就是成本的一部分,那么尽可能减少工人工资来提升净利润就理所当然了,在资本主义强势的时代下,这种状况更是屡见不鲜。

在实行“最低工资法”前,由于工资过低,众多基层工人正常生活都难以为继的例子不胜枚举;甚至在实行“最低工资法”后,仍然有为了增大利润将工人工资压至法律规定的底线的现象,下面一段取自解放网2010年6月8日的报道《富士康工资涨幅一周超十年:2000元起薪开拔》:

\begin{center}
\doublebox{\parbox{0.86\textwidth}{
\par\setlength\parindent{2em}富士康员工10年薪资水平的变动曲线正是这个“游戏规则”的一个缩影:基层员工工资总是“稳定”地保持在所在城市最低工资标准“高压红线”上,既不触犯,也不“超越”——有学者形象地将这种现象称为“贴线低空飞行”的“地板工资”。

\par\setlength\parindent{2em}工资薪水紧贴最低工资标准“符合法律,但没有道德”。

\par\setlength\parindent{2em}“这是合法的‘压榨’。”今年5月上旬,深圳富士康龙华园区“新干班”(是富士康培养管理干部的重要方式,其成员都是大学毕业生)成员张虎(化名)对早报记者说,虽然他的底薪并非基层流水线上作业员的950元的底薪,而是2000元/月。但提起工资,张虎还是不免发起了诸多“牢骚”:“那些基层的普工(即流水线上的)虽然每个月也能挣一千七八甚至两千元,但他们每个月的加班时间要超过100小时”;而作为储备干部的他们,即使加班,每个月的工资也基本上在2500-2800元之间,“很少超过3000元”。
}}
\end{center}

从这一篇报道,足可以看出,实行“最低工资法”的必要性与迫切性。

\section{“悖论性贫困”与资本强势的社会主义}
“悖论性贫困”的具体表现是:工人越是努力劳动,就越是贫困;工人劳动的生产力越是发展,就越是贫困。也就是说工人的贫困与其劳动、与其劳动生产力成正比关系。这种悖论性贫困是资本主义生产方式,即雇佣劳动制,造成的。

具有劳动能力的劳动者必须与生产资料,劳动条件相结合才能进行具有现实性的劳动,但是资本主义将二者分离开来,劳动者缺乏生产资料,就必须依赖于资本的雇佣才能进行生产,那么也就存在了无法通过劳动获得生活资料的风险,也就是存在失业的风险,从这个方面看来,造成失业的因素可能是多方面的,实行“最低工资法”并不一定导致失业人群增加,尤其是“低薪求职者再也找不到工作”,未免言过其实。资本为了保证利润,确实可能会做出调整,加剧求职者之间对相同岗位的竞争,这种状况多见于难以为继再生产的小微企业等,不过还要考虑到的是,法律(政治)是经济的一种反应,一个国家往往出台多个法律,彼此之间相互协调保证对社会秩序的维持,所以相应地,如我国出台了“最低工资规定”后,又出台了一系列扶持小微企业的政策,某种程度上讲,也是缓解了最低工资可能导致的求职者竞争加剧。

前面提到,雇佣劳动是死的劳动对活的劳动的单向度的、不平等的权力支配关系。而这种生产关系就是“悖论性贫困”的原因,劳动者通过劳动,为资本提供了剩余价值,也就是壮大了资本的力量,增强了资本对劳动者的权利支配关系。

在资本的发展过程中,为了获取更多利润,就要提高生产力,随着生产力发展,技术进步,机器生产体系的形成,资本的有机构成就会发生变化,价值构成中劳动者劳动力的价值不断减少,技术构成中必须的劳动量也不断减少,“机器本身增加生产者的财富,而它的资本主义应用使生产者变成需要救济的贫民”。由此,劳动力的价值就越得不到重视,因此劳动者的劳动能力相当于“贬值”了,就产生了“悖论性贫困”,劳动者现在越是努力工作,未来自己对资本的价值与吸引力就越低,就越难以获得使用生产资料的机会,面临失业的风险就越大,产生大量“过剩人口”,越来越多的劳动者难以获得生活资料,整个劳动者的力量就越薄弱。

而我们现在正处于“资本强势的社会主义”之中,资本的话语权强势,经济大失衡,收入差距悬殊,两极分化,“悖论性贫困”的影子也存在于我们的周围。我国作为社会主义国家,共同富裕才是我们的最终目标,所以法律必须要为底层劳动者发声,就像温家宝总理说的那样:“让劳动者活得有尊严。”我国出台了一系列法律法规和决议,目的就是抑制资本,保障劳动者权益。“最低工资规定”就是其中之一,而这一规定引起的一些非议,或许也从一个侧面反映了它确实发挥了它应有的作用,削弱资本的力量,重新分配社会财富,让底层劳动者能有法可依地维权,有尊严地生活。


\begin{thebibliography}{123456} 
\bibitem{ref1} 《最低工资法不可取》,薛兆丰,2001-4-2.
\bibitem{ref2} 《富士康工资涨幅一周超十年:2000元起薪开拔》,解放网,2010-06-08.
\bibitem{ref3} 《为什么生产力越是发展,工人越是努力劳动,他就越是贫困》,王峰明,2018-06-02.
\bibitem{ref4} 百度百科:“马克思的工资决定理论”词条。
\end{thebibliography}


\end{document}
